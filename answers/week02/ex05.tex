\begin{exercise}
Pruebe que toda función analítica de módulo constante es constante.
\end{exercise}

\begin{solution}
Sea $u + iv = f(x + iy)$ una función analítica, de módulo constante $u^2 + v^2 = r^2$, definida en un dominio conexo $U \subset \C$. Si $r = 0$, entonces necesariamente $f = 0$ es la función nula, que es constante. Por ello, en adelante asumiremos que $r > 0$.

Diferenciando la premisa $u^2 + v^2 = r^2$ con respecto a $x, y$, tenemos
$$u \p ux + v \p vx = 0, \qquad \qquad u \p uy + v \p vy = 0$$
Sustituyendo las condiciones de Cauchy-Riemann en estas ecuaciones, tenemos
$$u \p vy - v \p uy = 0, \qquad \qquad -u \p vx + v \p ux = 0$$
Puesto $u, v$ no se anulan simultáneamente en $U$, la única solución de estas cuatro ecuaciones es
$$\p ux = \p uy = \p vx = \p vy = 0$$
Por ende, $u, v$ son constantes. Por ende, $f$ es constante.
\end{solution}
