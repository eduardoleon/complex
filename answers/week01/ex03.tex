\begin{exercise}
Pruebe que en $\C$ no existe el conjunto de los números positivos.
\end{exercise}

\begin{remark}
Decimos que un subconjunto $P \subset F$ de un cuerpo está formado por elementos positivos si satisface: (a) $0 \notin P$, (b) para cada $x \ne 0$, obtenemos $x \in P$ o bien $-x \in P$, (c) $P$ es cerrado bajo las dos operaciones del cuerpo.
\end{remark}

\begin{solution}
Sea $P \subset \C$ un subconjunto que satisface (b) y (c). Puesto que $i \ne 0$, o bien $i \in P$ o bien $-i \in P$. Puesto que $C$ es cerrado bajo la multiplicación, $(\pm i)^2 = -1 \in P$, $(-1)^2 = 1 \in P$. Puesto que $P$ es cerrado bajo la suma, $(-1) + 1 = 0 \in P$. Por ende, $P$ no satisface (a).
\end{solution}
