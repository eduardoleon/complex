\begin{exercise}
Pruebe que el sistema $F$ de las matrices de la forma
$$
\mat {\alpha & \beta \\ {-\beta} & \alpha}
    = \alpha \mat {1 & 0 \\ 0 & 1} + \beta \mat {0 & 1 \\ {-1} & 0}
    = \alpha I + \beta J, \qquad \alpha, \beta \in \R
$$
satisface las siguientes propiedades:
\begin{enumerate}
    \item $F$ es un cuerpo.
    \item $F$ contiene un subcuerpo isomorfo a $\R$.
    \item La ecuación $x^2 + 1 = 0$ tiene solución en $F$.
\end{enumerate}
\end{exercise}

\begin{solution}
\leavevmode
\begin{enumerate}[label=(\alph*)]
    \item Verifiquemos que $F$ es cerrado bajo la adición:
    $$(\alpha I + \beta J) + (\gamma I + \delta J) = (\alpha + \gamma) I + (\beta + \delta) J$$
    
    Verifiquemos que $F$ es cerrado bajo la multiplicación:
    $$(\alpha I + \beta J) (\gamma I + \delta J) = (\alpha \gamma - \beta \delta) I + (\alpha \delta + \beta \gamma) J$$
    
    Verifiquemos que todo elemento no nulo de $F$ tiene inversa multiplicativa en $F$:
    $$(\alpha I + \beta J)^{-1} = \frac \alpha {\alpha^2 + \beta^2} I - \frac \beta {\alpha^2 + \beta^2} J$$
    
    Por ende, $F$ es un cuerpo.
    
    \item Por simple inspección, la aplicación $\varphi : \R \to F$, $\varphi(x) = xI$ es un homomorfismo de anillos inyectivo. Por ende, la imagen de $\varphi$ es un subcuerpo de $F$ isomorfo a $\R$.
    
    \item Sustituyendo $x = J$, tenemos $x^2 + 1 = J^2 + I = -I + I = 0$.
\end{enumerate}
\end{solution}
