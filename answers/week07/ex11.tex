\begin{exercise}
Pruebe las siguientes afirmaciones:
\begin{enumerate}[label=(\alph*)]
    \item Sea $f(z)$ es una función continua en una región $U \subset \C$ tal que la integral de contorno
    $$\int_{\partial R} f(z) \, dz = 0$$
    se anula para todo rectángulo cerrado $R \subset U$. Entonces $f(z)$ es analítica en $U$.
    
    \item Sea $f(z)$ es una función analítica en una región $U \subset \C$. Entonces $f(z)$ tiene derivadas de todos los órdenes y todas ellas son funciones analíticas.
    
    \item Toda función entera y acotada es constante.
    
    \item Todo polinomio no constante tiene por lo menos una raíz.
\end{enumerate}
\end{exercise}

\begin{solution}
\leavevmode
\begin{enumerate}[label=(\alph*)]
    \item Puesto que la analiticidad es una propiedad local, supondremos sin pérdida de generalidad que $f(z)$ está definida en una región convexa $U$.
    
    Para facilitar la discusión, daremos las siguientes definiciones:
    \begin{itemize}
        \item Un tramo es un segmento orientado, que puede ser horizontal o vertical.
        
        \item Un zigzag es una curva formada concatenando tramos.
        
        \item Una ``letra u'' es un zigzag de tres tramos $\upsilon = \lambda \star \mu \star \rho$ tal que
        \begin{itemize}
            \item $\lambda, \rho$ tienen la misma longitud y direcciones opuestas.
            \item $\lambda, \rho$ son perpendiculares a $\mu$.
            \item $\mu$ puede ser degenerado, i.e., tener longitud cero.
        \end{itemize}
        
        Si $\upsilon = \lambda \star \mu \star \rho$ es una ``letra u'', entonces
        \begin{itemize}
            \item $\upsilon'$ denotará el tramo paralelo a $\mu$ que conecta los extremos de $\upsilon$.
            \item $\upsilon' \star \upsilon^{-1}$ es la frontera de un rectángulo contenido en $U$, porque $U$ es convexo.
        \end{itemize}
        
        \item La complejidad de un zigzag $\gamma$ es el menor número de tramos $\sigma_1, \dots, \sigma_n$ con los cuales $\gamma$ puede ser reconstruido como $\gamma = \sigma_1 \star \dots \star \sigma_n$. Los caminos constantes tienen complejidad cero.
    \end{itemize}
    
    Sea $\gamma$ un zigzag cerrado. Tenemos dos posibilidades:
    \begin{itemize}
        \item Si $\gamma = \alpha \star \upsilon \star \beta$, donde $\alpha, \beta$ son subzigzags arbitrarios y $\upsilon$ es una ``letra u'', entonces
        $$\int_\upsilon f(z) \, dz = \int_{\upsilon'} f(z) \, dz$$
        Por construcción, $\gamma' = \alpha \star \upsilon' \star \beta$ menos complejo que $\gamma$, pero satisface
        $$\int_\gamma f(z) \, dz = \int_{\gamma'} f(z) \, dz$$
        
        \item Si $\gamma$ no contiene úes\footnote{Marcos Mundstock dixit: ``y, o, u, ae-ae''.}, entonces las partes real e imaginaria de $\gamma$ son funciones monótonas. Esto implica que $\gamma$ es un camino constante. De otro modo, $\gamma$ no regresaría al punto de partida.
    \end{itemize}
    
    Por inducción en la complejidad de $\gamma$, tenemos
    $$\int_\gamma f(z) \, dz = 0$$
    Por ende, las integrales de línea de $f$ sobre zigzags son independientes de la trayectoria.
    
    Tomemos un punto arbitrario $z_0 \in U$. Definamos la función $F : U \to \C$ por
    $$F(z_1) = \int_\gamma f(z) \, dz$$
    donde $\gamma : [0,1] \to U$ es un zigzag con extremos en $\zeta(k) = z_k$ para $k = 0, 1$. Por construcción,
    $$\p Fx = -i \p Fy = f(z)$$
    Por ende, $f$ es localmente la derivada de una función analítica. La prueba se completa demostrando que la derivada de una función analítica es también analítica. Esto se hará en el siguiente ítem.
    
    \item Puesto que la analiticidad es una propiedad local, supondremos sin pérdida de generalidad que $f(z)$ está definida sobre una región simplemente conexa $U$.
    
    Definamos la $n$-ésima \textit{derivada formal} de una función analítica $f : U \to \C$ como
    $$f^{(n)}(z) = \frac {n!} {2\pi i} \int_\gamma \frac {f(\zeta)} {(\zeta - z)^{n+1}} \, d\zeta$$
    donde $\gamma \subset U$ es un lazo antihorario simple alrededor de $z_0 \in U$.
    
    Llamemos $U_0$ a la región $U$ agujereada en $z_0$. Haremos algunas verificaciones de rutina:
    \begin{itemize}
        \item El integrando en la definición de $f^{(n)}(z_0)$ es analítico en $U_0$. Entonces el valor de $f^{(n)}(z_0)$ sólo depende de la clase de homotopía de $\gamma$ en $U_0$. Esta clase está determinada por
        \begin{itemize}
            \item La hipótesis de que $U$ es simplemente conexo.
            \item El requerimiento de que $\gamma$ sea un lazo antihorario simple alrededor de $z_0$.
        \end{itemize}
        
        Por ende, $f^{(n)}$ es una función bien definida.
        
        \item Una vez fijada la curva $\gamma$, el punto $z_0$ tiene una vecindad $D \subset U$ encerrada por $\gamma$. Para $z \to z_0$, podemos asumir que $z \in D$. Además, la convergencia
        $$\frac {f(\zeta)} {(\zeta - z)^{n+1}} \to \frac {f(\zeta)} {(\zeta - z_0)^{n+1}}$$
        es uniforme en $\gamma$. Entonces, por el ejercicio 17 de la semana 6,
        $$
        \lim_{z \to z_0} f^{(n)}(z)
            = \frac {n!} {2\pi i} \int_\gamma \lim_{z \to z_0} \frac {f(\zeta)} {(\zeta - z)^{n+1}} \, d\zeta
            = \frac {n!} {2\pi i} \int_\gamma \frac {f(\zeta)} {(\zeta - z_0)^{n+1}} \, d\zeta
            = f^{(n)}(z_0)
        $$
        
        Por ende, $f^{(n)}$ es una función continua.
    \end{itemize}
    
    Dado un número natural $n \in \N$, llamemos $P(n)$ a la siguiente proposición:
    \begin{displayquote}
    ``Toda función analítica $f$ es $n$ veces diferenciable y, para todo $k = 0, 1, 2, \dots, n$, la $k$-ésima derivada de $f$ es igual a $f^{(k)}.$''
    \end{displayquote}
    
    Demostraremos $P(n)$ para todo $n \in \N$. Empecemos con el caso base $P(0)$. Por construcción,
    $$F(z) = \dfrac {f(z) - f(z_0)} {z - z_0}$$
    es una función analítica en $U_0$. Entonces la integral de contorno
    $$
    \int_\gamma F(z) \, dz
        = \int_\gamma \frac {f(z)} {z - z_0} \, dz
        - f(z_0) \cancelto {2\pi i} {\int_\gamma \frac {dz} {z - z_0}}
    $$
    sólo depende de la clase de homotopía de $\gamma$ en $U_0$.
    
    Por otro lado, la posibilidad de extender $F$ de manera continua a $z_0$, poniendo
    $$F(z_0) = \lim_{z \to z_0} F(z) = f'(z_0)$$
    implica que $F$ es acotada cerca de $z_0$. La cota local en módulo
    $$\left| \int_\gamma F(z) \, dz \right| \le V(\gamma) \cdot \Vert F \Vert_\infty$$
    implica que la integral puede hacerse arbitrariamente pequeña encogiendo $\gamma$. Por ende,
    $$\int_\gamma F(z) \, dz = 0 \implies f(z_0) = \frac 1 {2\pi i} \int_\gamma \frac {f(z)} {z - z_0} \, dz$$
    
    Pasemos al caso inductivo. Calculando mecánicamente, se verifica la identidad
    $$
    \frac 1 {(\zeta - z)^{n+1}} - \frac 1 {(\zeta - z_0)^{n+1}}
        = \frac 1 {(\zeta - z)^n (\zeta - z_0)}
        - \frac 1 {(\zeta - z_0)^{n+1}}
        + \frac {z - z_0} {(\zeta - z)^{n+1} (\zeta - z_0)}
    $$
    
    Pongamos $g(\zeta) = f(\zeta) / (\zeta - z_0)$. Entonces, para todo $z \in U_0$ encerrado por $\gamma$, tenemos
    $$f^{(n)}(z) - f^{(n)}(z_0) = ng^{(n-1)}(z) - ng^{(n-1)}(z_0) + g^{(n)}(z) \cdot (z - z_0)$$
    
    Dividiendo entre $z - z_0$ y tomando el límite cuando $z \to z_0$, tenemos
    $$
    \lim_{z \to z_0} \frac {f^{(n)}(z) - f^{(n)}(z_0)} {z - z_0}
        = \lim_{z \to z_0} \frac {ng^{(n-1)}(z) - ng^{(n-1)}(z_0)} {z - z_0}
        + \lim_{z \to z_0} g^{(n)}(z)
    $$
    
    Asumiendo inductivamente $P(n)$, tenemos
    $$\lim_{z \to z_0} \frac {f^{(n)}(z) - f^{(n)}(z_0)} {z - z_0} = (n+1) g^{(n)}(z_0) = f^{(n+1)}(z_0)$$
    
    Esto es, $P(n+1)$. Por ende, $P(n)$ para todo $n \in \N$. Obviamente esto implica que las derivadas $f^{(n)}$ (ya no necesitamos el adjetivo ``formales'') son todas analíticas.
    
    \item Sea $f : U \to \C$ una función analítica definida en una región simplemente conexa $U \subset \C$ y sea $\gamma \subset U$ un círculo alrededor de un punto distinguido $z \in U$. Por la fórmula integral de Cauchy,
    $$|f^{(n)}(z)| \le \frac {n!} {2\pi} \int_\gamma |g_n(\zeta)| \, |d\zeta| \le \frac {n!} {2\pi} \cdot V(\gamma) \cdot \Vert g_n \circ \gamma \Vert_\infty = rn! \cdot \Vert g_n \circ \gamma \Vert_\infty$$
    donde $g_n(\zeta) = f(\zeta) / (\zeta - z)^{n+1}$. En particular,
    \begin{itemize}
        \item Si $f$ es entera, entonces $z \in \C$, $r > 0$ pueden ser arbitrarios.
        \item Si $f$ está acotada por $C > 0$, entonces $\Vert g_1 \circ \gamma \Vert_\infty < Cr^{-2}$.
    \end{itemize}
    
    Si $f$ es entera y acotada, entonces $|f'(z)| < Cn! r^{-1}$ con $r > 0$ arbitrario. Esto implica que $f'(z) = 0$. Puesto que $z \in \C$ también es arbitrario, tenemos $f' = 0$, lo cual implica que $f$ es constante.
    
    \item Sea $P(z)$ un polinomio no constante y sin raíces. Entonces,
    
    \begin{itemize}
        \item Puesto que $P(z)$ es no constante $|P(z)| \to \infty$ cuando $|z| \to \infty$.
        \item Puesto que $P(z)$ no tiene raíces, la recíproca $1/P(z)$ es una función entera.
        \item Puesto que $|P(z)| \to \infty$ cuando $|z| \to \infty$, la recíproca $1/P(z)$ es una función acotada.
    \end{itemize}
    
    Entonces $1/P(z)$ es constante. Por ende, $P(z)$ era constante después de todo.
\end{enumerate}
\end{solution}
