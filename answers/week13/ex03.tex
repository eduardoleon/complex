\begin{exercise}
Desarrolle la función $f(z)$ definida por
$$f(z) = \frac 1 {1 + z^2}$$
en potencias de $(z - 1)$.
\end{exercise}

\begin{solution}
Los polos de $f(z)$ son $\alpha^2, \alpha^6$, donde $\alpha = e^{i\pi/4}$. Poniendo $z = 1 + w \sqrt 2$, tenemos
$$
f(z)
    = \frac {i/2} {w \sqrt 2 + 1 + i} + \frac {-i/2} {w \sqrt 2 + 1 - i}
    = \frac 1 {2 \sqrt 2} \left[
        \frac {\alpha^2}    {w + \alpha} +
        \frac {\alpha^{-2}} {w + \alpha^{-1}} \right]
$$
Expandiendo la serie de potencias centrada en $w = 0$, tenemos
$$
f(z)
    = \frac 1 {2 \sqrt 2} \left[
        \frac \alpha        {1 + \alpha^{-1} w} +
        \frac {\alpha^{-1}} {1 + \alpha      w} \right]
    = \sum_{n=0}^\infty \frac {\alpha^{1-n} + \alpha^{n-1}} {\sqrt {2^{n+3}}} \, (1 - z)^n
    = \sum_{n=0}^\infty \beta_n (z - 1)^n
$$
Entonces los coeficientes $\beta_n$ son
$$
\beta_{4k} = \frac {(-1)^k} {2^{2k+1}}, \qquad
\beta_{4k+1} = \frac {(-1)^{k+1}} {2^{2k+1}}, \qquad
\beta_{4k+2} = \frac {(-1)^k} {2^{2k+2}}, \qquad
\beta_{4k+3} = 0
$$
\end{solution}
