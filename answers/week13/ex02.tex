\begin{exercise}
Pruebe que la serie de potencias
$$\zeta(z) = \sum_{n=1}^\infty n^{-z}$$
converge en la región $\Re(z) > 1$.
\end{exercise}

\begin{solution}
Sea $x = \Re(z)$. Entonces,
$$
|\zeta(z)|
    = \left| \sum_{n=1}^\infty n^{-z} \right|
    \le \sum_{n=1}^\infty \left| n^{-z} \right|
    = \sum_{n=1}^\infty n^{-x}
    = \zeta(x)
$$
Para $n = \lceil t \rceil$ y $x > 1$, tenemos $t^{-x} \ge n^{-x}$, así que
$$
\zeta(x) - 1
    = \sum_{n=2}^\infty n^{-x}
    \le \int_1^\infty t^{-x} \, dt
    = \frac 1 {x-1}
    < \infty
$$
Por ende, $\zeta(z)$ es absolutamente convergente en el semiplano $\Re(z) > 1$. De hecho, para $x \ge 1 + \varepsilon$, tenemos la cota uniforme $1/\varepsilon$, así que $\zeta(z)$ es absoluta y uniformemente convergente en el semiplano $\Re(z) \ge 1 + \varepsilon$.
\end{solution}
