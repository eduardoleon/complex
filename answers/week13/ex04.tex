\begin{exercise}
Utilice la representación en una serie de Taylor para probar que el conjunto de polos de una función definida en una región no tiene puntos de acumulación en dicha región.
\end{exercise}

\begin{solution}
Sea $f(z)$ una función meromorfa en una región $U \subset \C$ y supongamos por el absurdo que $z_0 \in U$ es un punto de acumulación de los polos de $f(z)$. Entonces,
\begin{itemize}
    \item Si $f(z)$ es analítica en $z_0$, entonces la serie de Taylor
    $$f(z_0) + f'(z_0) \, (z - z_0) + \frac {f''(z_0)} 2 \, (z - z_0)^2 + \dots$$
    converge en un disco centrado en $z_0$. Esto es imposible, porque existe una sucesión de polos de $f(z)$ que converge a $z_0$.
    
    \item Si $f(z)$ tiene un polo en $z_0$, entonces $g(z) = 1/f(z)$ se anula en $z_0$ y su serie de Taylor
    $$\frac {g^{(m)}(z_0)} {m!} \, (z - z_0)^m + \frac {g^{(m+1)}(z_0)} {(m+1)!} \, (z - z_0)^{m+1} + \dots$$
    converge en un disco centrado en $z_0$. Dividiendo entre $(z - z_0)^m$, la serie
    $$\frac {g^{(m)}(z_0)} {m!} + \frac {g^{(m+1)}(z_0)} {(m+1)!} \, (z - z_0) + \dots$$
    converge a una función analítica que no se anula en una vecindad de $z_0$. Esto también es imposible, porque existe una sucesión de ceros de $g(z)$ que converge a $z_0$.
\end{itemize}
\end{solution}
