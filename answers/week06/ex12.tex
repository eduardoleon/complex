\begin{exercise}
Decimos que un camino suave por tramos $\gamma : [a,b] \to \C$ es rectificable si tanto su parte real $\Re \circ \gamma$ como su parte imaginaria $\Im \circ \gamma$ son de variación acotada. Decimos que otro camino rectificable $\sigma$ es equivalente a $\gamma$ si se puede expresar como una reparametrización $\sigma = \gamma \circ \varphi$, donde $\varphi : [c,d] \to [a,b]$ es una función suave y estrictamente creciente.

Una curva rectificable es una clase de equivalencia de caminos rectificables. Muestre que la integral de una función continua $f : U \to \C$ sobre una curva rectificable está bien definida. Esto es,
$$\int_\gamma f(z) \, dz = \int_\sigma f(z) \, dz$$
para todo par de caminos rectificables equivalentes $\gamma, \sigma$.
\end{exercise}

\begin{solution}
Sea $\P(\gamma)$ el conjunto de particiones del dominio de $\gamma$. Para todo $P \in \P(\gamma)$, abreviemos
$$S_f(P) = \sum_{k=1}^n f \circ \gamma(\tau_k) \cdot [\gamma(t_k) - \gamma(t_{k-1})]$$
donde $\tau_k \in [t_{k-1}, t_k]$ es arbitrario. Abreviemos también
$$S_f(\gamma) = \int_\gamma f(z) \, dz$$
Dado un margen de error $\varepsilon > 0$, existe $\delta > 0$ tal que, para todo $P \in \P(\gamma)$, se cumple
$$\Vert P \Vert < \delta \implies |S_f(\gamma) - S_f(P)| < \varepsilon/2$$
Análogamente, existe $\delta' > 0$ tal que, para todo $Q \in \P(\sigma)$, se cumple
$$\Vert Q \Vert < \delta' \implies |S_f(\sigma) - S_f(Q)| < \varepsilon/2$$
Puesto que los caminos $\gamma, \sigma$ son parametrizadaos por intervalos compactos, el cambio de parámetro $\varphi$ que los relaciona es una función uniformemente continua. Reduciendo $\delta'$, podemos suponer que
$$\Vert Q \Vert < \delta' \implies \Vert \varphi(Q) \Vert < \delta$$
donde $\varphi(Q) \in \P(\gamma)$ es la partición cuyos tramos son las imágenes $\varphi(T)$ de cada tramo $T$ de $Q$. Entonces, por construcción, $S_f(Q) = S_f(\varphi(Q))$. Tomando cualquier $\Vert Q \Vert < \delta'$, obtenemos
$$
|S_f(\gamma) - S_f(\sigma)|
    \le |S_f(\gamma) - S_f(\varphi(Q))| + |S_f(Q) - S_f(\sigma)|
    \le \varepsilon/2 + \varepsilon/2
    = \varepsilon
$$
Puesto que $\varepsilon$ es arbitrariamente pequeño, tenemos $S_f(\gamma) = S_f(\sigma)$.
\end{solution}
