\begin{exercise}
Sea $g : [a,b] \to \R$ una función arbitraria. Dada una partición $P : a = t_0 < \dots < t_n = b$ del intervalo $[a, b]$, considere la siguiente estimación:
$$V_g(P) = \sum_{k=1}^n |g(t_k) - g(t_{k-1})|$$
La variación total de $g$ en $[a,b]$ se define como
$$V_g([a,b]) = \sup_P V_g(P)$$
Decimos que $g$ es de variación acotada si $V_g([a, b]) < \infty$.

\newpage

\noindent Pruebe las siguientes afirmaciones:
\begin{enumerate}[label=(\alph*)]
    \item Toda función creciente $f : [a,b] \to \R$ tiene variación total $V_f([a,b]) = f(b) - f(a)$.
    
    \item Una función $g$ es de variación acotada si y sólo si existe una función creciente $f$ tal que, para todo $x < y$, el incremento $g(y) - g(x)$ está acotado superiormente por $f(y) - f(x)$.
    
    \item La función $g : [0,1] \to \R$ definida por $g(0) = 0$, $g(x) = x \sin(1/x)$ para $x \ne 0$, es continua y acotada, pero no de variación acotada. No obstante, $g(x) = x \, f(x)$ sí es de variación acotada.
    
    \item Sea $g : [a,b] \to \R$ una función de variación acotada y sea $f(x) = V_g([a,x])$. Entonces $g, f$ comparten los mismos puntos de continuidad y discontinuidad.
\end{enumerate}
\end{exercise}

\begin{solution}
\leavevmode
\begin{enumerate}[label=(\alph*)]
    \item Sin importar cómo tomemos la partición $P$, siempre tendremos
    $$V_f(P) = \sum_{k=1}^n [f(t_k) - f(t_{k-1})] = f(t_n) - f(t_0) = f(b) - f(a)$$
    Esto implica que $V_f([a,b]) = f(b) - f(a)$.
    
    \item Supongamos que $g$ es de variación acotada. Pongamos $f(x) = V_g([a,x])$. Dados $x \le y$, tenemos
    $$f(y) - f(x) = V_g([a,y]) - V_g([a,x]) = V_g([x,y])$$
    porque toda partición de $[a,y]$ puede ser refinada por otra partición que contiene a $x$. En particular, esto implica que $f$ es monótona. Tomando la partición trivial $P : a_0 = t_0 < t_1 = b$, tenemos
    $$g(y) - g(x) \le |g(y) - g(x)| = V_g(P) \le V_g([x, y]) = f(y) - f(x)$$
    
    Recíprocamente, supongamos ahora que $g$ admite una función creciente $f$ tal que $h = f-g$ también es creciente. Para toda partición $P$, tenemos
    $$|g(t_k) - g(t_{k-1})| \le |f(t_k) - f(t_{k-1})| + |h(t_k) - h(t_{k-1})|$$
    en cada subintervalo $[t_{k-1}, t_k]$. Entonces $V_g(P) \le V_f(P) + V_h(P)$. Tomando el supremo sobre todas las particiones, tenemos $V_g([a,b]) \le V_f([a,b]) + V_h([a,b])$. Por ende, $g$ es de variación acotada.
    
    \item Por construcción, $f$ está acotada en valor absoluto por la identidad. Esto implica, por el teorema de estricción, que $f$ es continua en $x = 0$. Por supuesto, $f$ también es continua lejos de $x = 0$.
    
    Sea $x_n$ la sucesión de múltiplos impares de $\pi/2$. Por construcción, tenemos
    $$f(1 / x_n) = \frac 1 {x_n} \sin(x_n) = \frac {(-1)^n} {x_n}$$
    Dos valores consecutivos de $f(1 / x_n)$ están separados por
    $$|f(1 / x_n) - f(1 / x_{n+1})| = \frac 1 {x_n} + \frac 1 {x_{n+1}}$$
    Obtenemos valores arbitrariamente grandes de $V_f(P)$ utilizando las particiones
    $$P : 0 < \frac 1 {x_n} < \dots < \frac 1 {x_0} < 1$$
    Por ende, $V_f([0,1]) = \infty$. Es decir, $f$ no es de variación acotada.
    
    Ahora analizaremos $g$. La derivada de $g$ lejos de $x = 0$ se calcula mecánicamente:
    $$g'(x) = 2x \sin(1/x) - \cos(1/x)$$
    Entonces $g'$ está acotada en valor absoluto lejos de $x = 0$ por
    $$|g'(x)| \le 2 \, |x \sin(1/x)| + \lvert \cos(1/x) \rvert \le 2 + 1 = 3$$
    Puesto que $g$ es continua en $0$, también tenemos
    $$|g(a) - g(0)| = \lim_{x \to 0} |g(a) - g(x)| \le \lim_{x \to 0} 3 |a - x| = 3a$$
    Entonces $3$ es una constante de Lipschitz para $g$. Dada una partición arbitraria $P$, tenemos
    $$V_g(P) = \sum_{k=1}^n |g(t_k) - g(t_{k-1}| \le \sum_{k=1}^n 3 \, (t_k - t_{k-1}) = 3$$
    Por ende, $V_g([0,1]) \le 3$. Es decir, $g$ es de variación acotada.
    
    \item Supongamos que $g$ es continua en $b$. Dado un margen de error $\varepsilon > 0$, tomemos una partición $P$ del intervalo $[a,b]$ que cumpla las siguientes condiciones:
    \begin{itemize}
        \item El error de estimación $\Delta(P) = f(b) - V_g(P)$ no excede $\varepsilon/2$.
        \item El tramo final $T$ satisface $|g(b) - g(x)| \le \varepsilon/2$ para todo $x \in T$.
    \end{itemize}
    
    Dado un punto $x \in T$, llamemos $P_x$ al refinamiento de $P$ con el punto adicional $x$ y llamemos $T_x$ al tramo final de $P_x$. Por supuesto, $\Delta(P_x) \le \varepsilon/2$. Entonces,
    $$f(b) - f(x) = V_g(T_x) \le V_g(P_x \cap T_x) + \varepsilon/2 = |g(b) - g(x)| + \varepsilon/2 = \varepsilon$$
    Por ende, $f$ también es continua en $b$.
    
    Supongamos ahora que $g$ no es continua en $b$. Entonces existe una sucesión $x_n$ que converge a $b$ tal que cada $g(x_n)$ está separado de $g(b)$ por un margen uniforme $\varepsilon > 0$. Luego,
    $$f(b) - f(x_n) = V_g([x_n, b]) \ge |g(b) - g(x_n)| \ge \varepsilon$$
    Por ende $f$ tampoco es continua en $b$.
\end{enumerate}
\end{solution}
