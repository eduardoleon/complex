\begin{exercise}
Sea $u(z)$ una función armónica en un disco $D \subset \C$ centrado en $z_0 \in D$. Pruebe que
$$v(z_1) = \int_{\gamma} \p ux \, dy - \p uy \, dx$$
es una armónica conjugada de $u(z)$ en $D$, donde $\gamma$ es el segmento orientado desde $z_0$ hasta $z_1$.
\end{exercise}

\begin{solution}
Puesto que $u(z)$ es armónica, el integrando
$$\p ux \, dy - \p uy \, dx$$
es una forma cerrada. Puesto que $D$ es simplemente conexo, el integrando es exacto. Así pues, la integral que define a $v(z_1)$ puede calcularse a lo largo de cualquier trayectoria desde $z_0$ hasta $z_1$, sin alteración del resultado. Tomando $\gamma$ cuyo último tramo es horizontal o vertical, tenemos, respectivamente,
$$\p vx = -\p uy, \qquad \qquad \qquad \p vy = \p ux$$
Por ende, $v(z)$ es una armónica conjugada de $u(z)$.
\end{solution}
