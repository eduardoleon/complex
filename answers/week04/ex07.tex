\begin{exercise}
Pruebe que $A$ es abierto (cerrado) en $\C$ si y sólo si es igual a su interior (clausura).
\end{exercise}

\begin{solution}
Las siguientes proposiciones son equivalentes:
\begin{itemize}
    \item $A$ es abierto (cerrado) en $\C$.
    \item $A$ es un abierto (cerrado) de $\C$ contenido en (que se contiene a) sí mismo.
    \item $A$ es el mayor abierto (menor cerrado) de $\C$ contenido en (que se contiene a) sí mismo.
    \item $A$ es el interior (la clausura) de sí mismo en $\C$.
\end{itemize}
\end{solution}
