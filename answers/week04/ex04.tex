\begin{exercise}
Considere el conjunto $A \subset \C$ de los números complejos cuyas partes real e imaginarias son racionales. Describa el interior, la clausura y la frontera de $A$.
\end{exercise}

\begin{solution}
Por construcción $A$ es el subconjunto $\Q^2$ del plano $\C = \R^2$. Entonces $A$ es numerable. Por ende el interior de $A$ es vacío: ningún otro abierto de $\C$ se puede encajar en $A$.

Por otro lado, todo $x \in \R$ es el límite de una sucesión $x_n \in \Q$. Entonces todo $x + iy \in \C$ es el límite de una sucesión $x_n + iy_n \in A$. Por ende la clausura de $A$ es todo $\C$.

Utilizando los dos resultados anteriores, la frontera de $A$ es su clausura $\C$ menos su interior $\varnothing$, i.e., la frontera de $A$ es todo el plano $\C$.
\end{solution}
