\begin{exercise}
Calcule la serie de potencias de
$$f(z) = \frac {2z + 3} {z + 1}$$
centrada en $z = 1$ y determine su radio de convergencia.
\end{exercise}

\begin{solution}
Pongamos $w = z - 1$ para simplificar los cálculos. Entonces,
$$
\tilde f(w)
    = \frac {2w + 5} {w + 2}
    = 2 + \frac 1 {w + 2}
    = 2 + \frac {1/2} {1 + w/2}
    = 2 + \frac 12 \sum_{n=0}^\infty \left ( -\frac w2 \right)^n
    = \frac 52 + \sum_{n=1}^\infty \frac {(-1)^n w^n} {2^{n+1}}
$$
Regresando a la variable original, tenemos
$$f(z) = \frac 52 + \sum_{n=1}^\infty \frac {(-1)^n (z-1)^n} {2^{n+1}} = \sum_{n=0}^\infty a_n$$
Consideremos la cantidad auxiliar
$$
\lim_{n \to \infty} \left| \frac {a_{n+1}} {a_n} \right|
    = \lim_{n \to \infty} \frac {2^n} {2^{n+1}} \cdot |z-1|
    = \frac {|z-1|} 2
$$
Esta serie converge absolutamente cuando $|z-1| < 2$ y no converge cuando $|z-1| > 2$. Por ende, el radio de convergencia de la serie es $2$.
\end{solution}
