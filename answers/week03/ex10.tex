\begin{exercise}
Determine los valores de $z \in \C$ para los cuales la serie
$$\sum_{n=0}^\infty \left( \frac z {1+z} \right)^n$$
es convergente.
\end{exercise}

\begin{solution}
La serie dada es una serie geométrica. La condición necesaria y suficiente para que esta serie sea convergente es que la razón entre dos términos consecutivos tenga módulo menor que $1$. Esto es,
$$\left| \frac z {1+z} \right| < 1 \iff |z| < |1+z| \iff |z|^2 < |1+z|^2$$
Pongamos $z = x + iy$. Entonces $|z|^2 = x^2 + y^2$, $|1+z|^2 = (1+x)^2 + y^2$. Entonces,
$$|z|^2 < |1+z|^2 \iff x^2 < (1+x)^2 \iff x > -1/2$$
Por ende, la serie dada converge para $\Re(z) > -1/2$ y no converge para $\Re(z) \le -1/2$.
\end{solution}
