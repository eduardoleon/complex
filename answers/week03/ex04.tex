\begin{exercise}
Analice la convergencia y la convergencia absoluta de la serie
$$\sum_{n=1}^\infty \frac {(-1)^n} {n+i}$$
\end{exercise}

\begin{solution}
Desdoblemos cada término de la serie en sus partes real e imaginaria:
$$
\frac {(-1)^n} {n+i} \cdot \frac {n-i} {n-i} =
\frac {(-1)^n n} {n^2 + 1} + i \frac {(-1)^{n+1}} {n^2 + 1}
$$
Las series de partes reales e imaginarias
$$
\sum_{n=1}^\infty \frac {(-1)^n n} {n+i}, \qquad \qquad
\sum_{n=1}^\infty \frac {(-1)^{n+1}} {n^2 + 1}
$$
son ambas convergentes, porque sus términos tienen signos alternantes y valores absolutos que convergen monótonamente a cero. Por ende, la serie del enunciado también es convergente. Sin embargo, la serie de módulos está acotada inferiormente por la serie armónica
$$
\sum_{n=1}^\infty \left| \frac {(-1)^n} {n+i} \right|
    = \sum_{n=1}^\infty \frac 1 {|n+i|}
    > \sum_{n=1}^\infty \frac 1 {n+1}
    > \int_2^\infty \frac 1x \, dx
    = \log x \Big \vert_2^\infty
= +\infty
$$
que es famosamente divergente. Por ende, la serie del enunciado no es absolutamente convergente.
\end{solution}
