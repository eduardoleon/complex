\begin{exercise}
Pruebe que toda sucesión de Cauchy es acotada.
\end{exercise}

\begin{solution}
Sea $x_n$ una sucesión de Cauchy en un espacio métrico. Por definición, existe un instante $k \in \N$ tal que las distancias $d(x_m, x_n)$ están acotadas por $1$ para todo $m, n \ge k$. Pongamos $c_n = d(x_n, x_k)$. Por construcción, $c_n$ está acotado por $1$ para todo $n \ge k$. Entonces $c_n$ está acotado por $r = \max(1, c_1, \dots, c_k)$ para todo $n \in \N$.

No estamos obligados a usar $x_k$ como referencia para el cálculo de las distancias. Si $a$ es cualquier otro punto del espacio, entonces $d(a, x_n)$ está acotado por $d(a, x_k) + r$ para todo $n \in \N$. En particular, si $x_n$ es una sucesión de Cauchy en un espacio vectorial normado, entonces la norma $\Vert x_n \Vert$ es la distancia al origen y está acotada por $\Vert x_k \Vert + r$ para todo $n \in \N$.
\end{solution}
