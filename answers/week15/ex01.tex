\begin{exercise}
Pruebe que, si cada $f \in \F$ tiene parte real positiva, entonces $\F$ es normal.
\end{exercise}

\begin{solution}
Consideremos la familia $\G = \{ \varphi \circ f : f \in \F \}$, donde $\varphi$ es la transformación de Möbius
$$\varphi(z) = \frac {z-1} {z+1}$$
Por construcción, todo $g \in \G$ toma valores en el disco unitario, así que $\Vert g|_K \Vert_\infty \le 1$ para todo subconjunto compacto $K \subset U$. Entonces $\G$ es una familia normal. Como la normalidad es una propiedad topológica, es invariante bajo homeomorfismos, así que $\F$ también es una familia normal.
\end{solution}
