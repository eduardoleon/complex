\begin{exercise}
Pruebe que, si $f$ tiene exactamente $m$ ceros en $U$, entonces $f_n$ tiene por lo menos $m$ ceros para todo $n \in \N$ suficientemente grande.
\end{exercise}

\begin{solution}
Sea $z_0 \in U$ un cero de $f$ y tomemos un pequeño círculo $\gamma \subset U$ alrededor de $z_0$. Descartando un prefijo de la sucesión $f_n$, podemos suponer que $Z(f_n, \gamma) = Z(f, \gamma)$ para todo $n \in \N$. Adicionalmente, cada $f_n$ podría tener otros ceros alejados de los ceros de $f$. Por ende, cada $f_n$ tiene por lo menos $m$ ceros.
\end{solution}
