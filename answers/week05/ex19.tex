\begin{exercise}
Verifique que el lado interior del círculo $|z-a| = R$ es el disco $|z-a| < R$.
\end{exercise}

\begin{solution}
Tomemos los siguientes puntos de referencia: $z_2 = a + R$, $z_3 = a + iR$, $z_4 = a - iR$. Es fácil ver que estos puntos recorren el círculo dado de manera antihoraria. Entonces la transformación
$$T(z) = \frac {z - z_3} {z - z_4} \cdot \cancelto i {\frac {z_2 - z_4} {z_2 - z_3}}$$
reproduce la orientación absoluta de dicho círculo. Poniendo $z = a + r$, tenemos
$$T(z)
    = i \cdot \frac {r - iR} {r + iR} \cdot \red {\frac {\bar r - iR} {\bar r - iR}}
    = \frac {R (r + \bar r) + i (r \bar r - R^2)} {\text {positivo}}
$$
Por ende, las siguientes proposiciones son equivalentes:
\begin{itemize}
    \item $z$ está en el lado interior (i.e. izquierdo) del círculo $|z-a| = R$.
    \item $T(z)$ tiene parte imaginaria negativa.
    \item $r \bar r - R^2$ es negativo.
    \item $r = z-a$ tiene módulo menor que $R$.
\end{itemize}
\end{solution}
