\begin{exercise}
Muestre que cualquier transformación de Möbius distinta de la identidad tiene a lo más dos puntos fijos.
\end{exercise}

\begin{solution}
Un punto fijo de $T$ no es otra cosa que un autovector de su representación matricial. Si $T$ tiene tres puntos fijos distintos, entonces su representación matricial tiene un único autoespacio de dimensión $2$, i.e., todo vector es un autovector, i.e., todo punto del plano extendido es un punto fijo.
\end{solution}
