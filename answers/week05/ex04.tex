\begin{exercise}
Pruebe que toda transformación de Möbius que fija el eje real (extendido, i.e., incluyendo el punto en el infinito) puede escribirse con coeficientes reales.
\end{exercise}

\begin{solution}
Recordemos que el plano extendido es, en realidad, la recta compleja proyectiva. Por ende, sus elementos se pueden describir como cocientes de números complejos. En particular, el punto en el infinito es un cociente con denominador cero.

Naturalmente, la recta real proyectiva es el subespacio de la recta compleja proyectiva formado por los cocientes de números reales. Si una transformación de Möbius $T$ fija la recta proyectiva real, entonces $T$ puede verse como un automorfismo lineal de $\R^2$, representado por una matriz real
$$T = \mat {a & b \\ c & d}$$
cuyas entradas se transcriben literalmente a la expresión
$$T(z) = \frac {az + b} {cz + d}$$
\end{solution}

\newpage
