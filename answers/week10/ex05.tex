\begin{exercise}
Sea $f(z)$ una función meromorfa en $U \subset \C$. Sean $a_1, \dots, a_m \in U$ los ceros y $b_1, \dots, b_n \in U$ los polos de $f(z)$, en ambos casos contados con multiplicidad. Pruebe que
$$
\frac 1 {2\pi i} \int_\gamma \frac {f'(z)} {f(z)} \, dz
    = \sum_{k=1}^m n(\gamma, a_k) - \sum_{k=1}^n n(\gamma, b_k)
$$
para todo ciclo $\gamma \subset U$ homólogo a cero que esquiva los ceros y polos de $f(z)$.
\end{exercise}

\begin{remark}
Este resultado se conoce como el \textit{principio del argumento} de Cauchy.
\end{remark}

\begin{solution}
Los polos de $f(z)$ pueden ser considerados ceros de orden negativo. Para justificar esta decisión, expresemos $f(z)$ alrededor de $z_0 \in U$ como $f(z) = (z - z_0)^r \, h(z)$, donde $h(z)$ es una función analítica que no se anula en $z_0$. Se verifica que
\begin{itemize}
    \item Si $r > 0$, entonces $z_0$ es un cero de $f(z)$ de orden $r$.
    \item Si $r < 0$, entonces $z_0$ es un polo de $f(z)$ de orden $-r$.
    \item Si $r = 0$, entonces $f(z_0) \in \C^\star$, así que $z_0$ no es ni cero ni polo de $f(z)$.
\end{itemize}
Entonces consideremos los siguientes objetos:
\begin{itemize}
    \item Los ceros y polos $z_1, \dots, z_s \in U$ de $f(z)$, contados \textit{sin} multiplicidad.
    \item La multiplicidad $r_k \in \Z$ de cada $z_k$, positiva si $z_k$ es un cero, negativa si $z_k$ es un polo.
    \item Un círculo pequeño $\gamma_k$ alrededor de cada $z_k$.
\end{itemize}
Por hipótesis, $\gamma$ es homóloga a $r_1 \gamma_1 + \dots + r_s \gamma_s$ en $U - \{ z_1, \dots, z_s \}$. Entonces,
$$
\int_\gamma \frac {f'(z)} {f(z)} \, dz
    = \sum_{k=1}^s n(\gamma, z_k) \int_{\gamma_k} \frac {f'(z)} {f(z)} \, dz
$$
Utilizando la expresión local $f(z) = (z - z_k)^{r_k} h_k(z)$, tenemos
$$
\int_{\gamma_k} \frac {f'(z)} {f(z)}
    = \int_{\gamma_k} \frac {r_k} {z - z_k}
    + \cancelto 0 {\int_{\gamma_k} \frac {h_k'(z)} {h_k(z)}}
    = 2\pi ir_k
$$
Sustituyendo en la sumatoria anterior, tenemos
$$\frac 1 {2\pi i} \int_\gamma \frac {f'(z)} {f(z)} \, dz = \sum_{k=1}^s r_k \cdot n(\gamma, z_k)$$
Separando los ceros ($m_k > 0$) y los polos ($m_k < 0$) y contándolos con multiplicidad, tenemos
$$
\frac 1 {2\pi i} \int_\gamma \frac {f'(z)} {f(z)} \, dz
    = \sum_{k=1}^m n(\gamma, a_k) - \sum_{k=1}^n n(\gamma, b_k)
$$
\end{solution}
