\begin{exercise}
Calcule las siguientes integrales:
\begin{enumerate}[label=(\alph*)]
    \item $S = \displaystyle \int_{-\infty}^\infty \frac {dx} {x^4 + 1}$
    \item $S = \displaystyle \int_{-\infty}^\infty \frac {x^2} {x^4 + x^2 + 1} \, dx$
    \item $S = \displaystyle \int_{-\infty}^\infty \frac {x^2 - x + 2} {x^4 + 10x^2 + 9} \, dx$
    \item $S = \displaystyle \int_{-\infty}^\infty \frac {x^2} {(x^2 + 1)^3} \, dx$
    \item $S = \displaystyle \int_0^\infty \frac {\cos x - 1} {x^2} \, dx$
    \item $S = \displaystyle \int_0^{2\pi} \frac {\cos 2x} {1 - 2a \cos x + a^2} \, dx$, donde $a^2 < 1$.
    \item $S = \displaystyle \int_0^{2\pi} \frac {dx} {(a + \cos x)^2}$, donde $a > 1$.
\end{enumerate}
\end{exercise}

\begin{solution}
Sea $\gamma \subset \C$ la curva del ejercicio anterior. Para los cuatro primeros integrandos, la cota
$$
\left| \int_{\gamma_2} f(z) \, dz \right|
    \le \int_{\gamma_2} |f(z)| \, |dz|
    \le \pi R \cdot \Vert f \circ \gamma_2 \Vert_\infty
$$
tiende a cero cuando $R \to \infty$. Entonces,
$$
S
    = \int_{-\infty}^\infty f(x) \, dx
    \approx \int_{\gamma_1} f(z) \, dz
    \approx \int_\gamma f(z) \, dz
$$
que se puede calcular usando la fórmula integral de Cauchy.
\begin{enumerate}[label=(\alph*)]
    \item Los polos del integrando están situados en las potencias impares de $\alpha = e^{\pi i/4}$, i.e., las raíces cuartas de $-1$. Los residuos correspondientes están dados por
    $$\Res(f, z) = \frac 1 {4z^3} = -\frac z4$$
    
    Los polos encerrados por $\gamma$ son $\alpha, \alpha^3$. Por ende, la integral pedida es
    $$
    S
        = \int_\gamma \frac {dz} {z^4 + 1}
        = -2\pi i \cdot \frac {\alpha + \alpha^3} 4
        = -2\pi i \cdot \frac {i \sqrt 2} 4
        = \frac \pi {\sqrt 2}
    $$
    
    \item Los polos del integrando las potencias no reales de $\alpha = e^{\pi i/3}$, i.e., las raíces cuadradas de las raíces cúbicas no reales de $1$. Los residuos correspondientes son
    $$\Res(f, z) = \frac {z^2} {4z^3 + 2z} = \frac z {4z^2 + 2}$$
    
    Los polos encerrados por $\gamma$ son $\alpha$ y $\alpha^2 = -\alpha^{-1}$. Por ende, la integral pedida es
    $$
    S
        = \int_\gamma \frac {z^2} {z^4 + z^2 + 1} \, dz
        = 2\pi i \cdot \left[ \frac \alpha {4\alpha^2 + 2} - \frac \alpha {4 + 2\alpha^2} \right]
        = 2\pi i \cdot \frac {-i/2} {\sqrt 3}
        = \frac \pi {\sqrt 3}
    $$
    
    \item Los polos del integrando son $z = \pm i$, $z = \pm 3i$. Los residuos correspondientes son
    $$\Res(f, z) = \frac {z^2 - z + 2} {4z^3 + 20z}$$
    
    Los polos encerrados por $\gamma$ son $z = i$, $z = 3i$. Los residuos correspondientes son
    $$
    \Res(f, i) = \frac {i^2 - i + 2} {4i^3 + 20i} = -\frac {1 + i} {16}, \qquad \qquad
    \Res(f, 3i) = \frac {9i^2 - 3i + 2} {108i^3 + 60i} = \frac {3 - 7i} {48}
    $$
    
    Por ende, la integral pedida es
    $$
    S
        = \int_\gamma \frac {z^2 - z + 2} {z^4 + 10z^2 + 9} \, dz
        = 2\pi i \cdot \frac {-10i} {48}
        = \frac {5\pi} {12}
    $$
    
    \item Los polos del integrando son $z = \pm i$, pero sólo $z = i$ está encerrado por $\gamma$. Antes de evaluar la integral pedida, consideremos la función auxiliar
    $$g(z) = \frac {z^2} {(z+i)^3}$$
    
    Derivando la función auxiliar dos veces, tenemos
    $$g'(z) = \frac {2z} {(z+i)^3} - \frac {3z^2} {(z+i)^4}$$
    $$g''(z) = \frac 2 {(z+i)^3} - \frac {12z} {(z+i)^4} + \frac {12z^2} {(z+i)^5}$$
    
    Por la fórmula integral de Cauchy, la integral pedida es
    $$
    S
        = \int_\gamma \frac {g(z)} {(z-i)^3} \, dz = \pi i \cdot g''(i)
        = \pi i \left[ \frac 2 {(2i)^3} - \frac {12i} {(2i)^4} + \frac {-12} {(2i)^5} \right]
        = \pi i \cdot \left[ \frac i4 - \frac {3i} 4 + \frac {3i} 8 \right]
        = \frac \pi 8
    $$
\end{enumerate}
El quinto integrando no satisface que $2\pi R \cdot \Vert f \circ \gamma_2 \Vert_\infty$ tienda a cero para $R \to \infty$, así que debemos utilizar un método más sofisticado para calcular la integral:
\begin{enumerate}[label=(\alph*)]
    \setcounter {enumi} 4
    
    \item Utilizaremos integración partes con
    \begin{align*}
        u &= 1 - \cos x & v &= 1/x \\
        du &= \sin x & dv &= -dx/x^2
    \end{align*}
    
    Entonces la integral pedida se reescribe como
    $$
    S
        = \int_0^\infty \frac {\cos x - 1} {x^2} \, dx
        = \cancelto 0 {\frac {1 - \cos x} x \bigg \vert_{x=0}^\infty}
            -\int_0^\infty \frac {\sin x} x \, dx
    $$
    
    El sumando cancelado se anula porque
    $$\lim_{x \to 0} \frac {1 - \cos x} x = \lim_{x \to \infty} \frac {1 - \cos x} x = 0$$
    
    Puesto que $\sin(x)/x$ es una función par, tenemos
    $$S = -\int_{-\infty}^0 \frac {\sin x} x \, dx = -\int_0^\infty \frac {\sin x} x \, dx$$
    
    Pongamos $q(z) = 1/z$, $f(z) = q(z) \, e^{iz}$. Por el lema de Jordan, tenemos la cota superior en módulo
    $$\left| \int_{\gamma_2} f(z) \, dz \right| \le \frac \pi n \cdot \Vert q \circ \gamma_2 \Vert_\infty$$
    que tiende a cero cuando $R \to \infty$. Entonces,
    $$
    -2S
        = \int_{-\infty}^\infty \Im \circ f(x) \, dx
        \approx \Im \circ \prv \left[ \int_{\gamma_1} f(z) \, dz \right]
        \approx \Im \circ \prv \left[ \int_\gamma f(z) \, dz \right]
    $$
    
    La curva $\gamma$ pasa por un polo de $f(z)$ en $z = 0$ y no encierra propiamente a ningún polo de $f(z)$. Por ende, el valor de la integral pedida es
    $$S = -\frac 12 \, \Im \big[ i\pi \cdot \Res(f, 0) \big] = -\frac \pi 2$$
\end{enumerate}
Finalmente, para evaluar las dos últimas integrales, consideremos el círculo unitario $\gamma \subset \C$, parametrizado como $z = e^{ix}$. Entonces, $dz = iz \, dx$, $2\cos x = z + z^{-1}$, $2\cos 2x = z^2 + z^{-2}$. Por ende,
\begin{enumerate}[label=(\alph*)]
    \setcounter {enumi} 5
    
    \item La integral pedida se reescribe como
    $$
    S
        = \int_\gamma \frac {z^2 + z^{-2}} {1 + a^2 - az - az^{-1}} \, \frac {dz} {2iz}
        = \frac i2 \int_\gamma \frac {z^4 + 1} {z^2 (z - a) (az - 1)} \, dz
        = \frac i2 \int_\gamma f(z) \, dz
    $$
    
    El integrando tiene dos polos encerrados por $\gamma$: uno simple en $z = a$, cuyo residuo es
    $$\Res(f, a) = \frac {a^4 + 1} {a^2 (a^2 - 1)}$$
    y uno doble en el origen. Consideremos la función auxiliar $g(z)$ y su derivada dadas por
    $$
    g(z) = \frac {z^4 + 1} {(z - a) (az - 1)}, \qquad \qquad
    g'(z) = g(z) \left[ \frac {4z^3} {z^4 + 1} - \frac 1 {z - a} - \frac a {az - 1} \right]
    $$
    
    Entonces el residuo en el origen es
    $$
    \Res(f, 0)
        = g'(0)
        = \frac 1a \left[ \frac 01 + \frac 1a + a \right]
        = \frac {1 + a^2} {a^2}
    $$
    
    Por ende, el valor de la integral pedida es
    $$
    S
        = -\pi \cdot [\Res(f, 0) + \Res(f, a)]
        = -\pi \left[ \frac {1 + a^2} {a^2} + \frac {a^4 + 1} {a^2 (a^2 - 1)} \right]
        = \frac {2\pi a^2} {1 - a^2}
    $$
    
    \item La integral pedida se reescribe como
    $$
    S
        = \int_\gamma \frac 4 {(2a + z + z^{-1})^2} \, \frac {dz} {iz}
        = \int_\gamma \frac {4z} {(z^2 + 2a + 1)^2} \, \frac {dz} i
    $$
    
    El integrando tiene dos polos dobles en $b_\pm = -a \pm \sqrt {a^2 - 1}$, pero sólo $b_+$ está encerrado por $\gamma$. Para hallar el residuo en $b_+$, consideremos la función auxiliar $g(z)$ y su derivada definidas por
    $$
    g(z) = \frac {4z} {(z - b_-)^2}, \qquad \qquad
    g'(z) = g(z) \left[ \frac 1z - \frac 2 {z - b_-} \right] = -4 \cdot \frac {z + b_-} {(z - b_-)^3}
    $$
    
    Entonces el valor de la integral pedida es
    $$
    S
        = \int_\gamma \frac {g(z)} {(z - b_+)^2} \, \frac {dz} i
        = 2\pi \cdot g'(b_+)
        = -8\pi \cdot \frac {b_+ + b_-} {(b_+ - b_-)^3}
        = \frac {2\pi a} {(a^2 - 1)^{3/2}}
    $$
\end{enumerate}
\end{solution}
