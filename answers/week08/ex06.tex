\begin{exercise}
Pruebe que las funciones $\exp$, $\sin$, $\cos$ tienen singularidades esenciales en el infinito.
\end{exercise}

\begin{solution}
Sea $U \subset \C$ una vecindad agujereada de $z = \infty$ en la esfera de Riemann.

\begin{enumerate}[label=(\alph*)]
    \item La prueba para $\exp$ ya fue dada en la parte (d) de la pregunta 1.
    
    \item Escribamos $\sin = f \circ \exp$, donde $f : \C^\star \to \C$ está definida por
    $$f(z) = \frac {z - z^{-1}} {2i}$$
    
    Tenemos $\sin(U) = f \circ \exp(U) = f(\C^\star)$. Para todo $a \in \C$, la ecuación cuadrática
    $$f(z) = a \implies z^2 - 2iaz - 1 = 0$$
    tiene por lo menos una solución distinta de cero. Entonces, $\sin(U) = f(\C^\star) = \C$.
    
    \item Escribamos $\cos = f \circ \exp$, donde $f : \C^\star \to \C$ está definida por
    $$f(z) = \frac {z + z^{-1}} 2$$
    
    Tenemos $\cos(U) = f \circ \exp(U) = f(\C^\star)$. Para todo $a \in \C$, la ecuación cuadrática
    $$f(z) = a \implies z^2 - 2az + 1 = 0$$
    tiene por lo menos una solución distinta de cero. Entonces, $\cos(U) = f(\C^\star) = \C$.
\end{enumerate}
\end{solution}
