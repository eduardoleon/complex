\begin{exercise}
Describa los ceros de la función $f(z) = \sin \dfrac {1 + z} {1 - z}$.
\end{exercise}

\begin{solution}
Consideremos la variable auxiliar
$$w = x + iy = \frac {1 + z} {1 - z}$$
Usando una conocida identidad trigonométrica, tenemos
$$\sin(x + iy) = \sin(x) \cos(iy) + \cos(x) \sin(iy) = \sin(x) \cosh(y) + i \cos(x) \sinh(y)$$
Entonces $\sin(w) = 0$ si y sólo si las siguientes condiciones se cumplen:
\begin{itemize}
    \item $\sin(x) \cosh(y) = 0$ implica $\sin(x) = 0$, porque $\cosh(y)$ no se anula para ningún $y \in \R$.
    \item $\cos(x) \sinh(y) = 0$ implica $\sinh(y) = 0$, porque $\sin(x) = 0$ implica $\cos(y) = \pm 1 \ne 0$.
    \item $\sin(x) = \sinh(y) = 0$ implica $w = n\pi$ para algún $n \in \Z$.
\end{itemize}
Invirtiendo la transformación de Möbius que pone a $w$ como función de $z$, tenemos
$$z = \frac {w - 1} {w + 1}$$
Los ceros de $f(z)$ se obtienen sustituyendo $w = n\pi$ en esta expresión. Para valores muy grandes de $n \in \Z$, sean positivos o negativos, obtenemos ceros de $f(z)$ cada vez más cercanos a $1$.
\end{solution}
