\begin{exercise}
Sea $f(z)$ una función analítica con una singularidad aislada en el punto $z = a$. Determine si las siguientes afirmaciones son equivalentes:
\begin{enumerate}[label=(\alph*)]
    \item $f$ tiene una singularidad removible en $z = a$.
    \item $f$ es acotada en una vecindad agujereada de $z = a$.
\end{enumerate}
\end{exercise}

\begin{solution}
Si la singularidad es removible, entonces $f$ admite una extensión que es analítica en $z = a$. Ante todo, dicha extensión debe ser continua, por ende acotada en una vecindad de $z = a$. Entonces $f$ también es acotada en una vecindad agujereada de $z = a$.

Recíprocamente, si $f$ está acotada en módulo en una vecindad de $z = a$, entonces la cota
$$\lim_{z \to a} |f(z) \cdot (z - a)| \le \Vert f \Vert_\infty \cdot\lim_{z \to a} |z - a| = 0$$
implica que la singularidad de $f$ en $z = a$ es removible. Por ende, las afirmaciones son equivalentes.
\end{solution}
