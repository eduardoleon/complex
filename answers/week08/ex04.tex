\begin{exercise}
Sea $f(z)$ una función analítica definida en la región $|z| > R$. Observe que esta región es una vecindad agujereada de $z = \infty$ en la esfera de Riemann. Entonces es geométricamente razonable decir que $f(z)$ tiene una singularidad aislada en $z = \infty$.

Para estudiar esta singularidad, usamos el cambio de carta $w = 1/z$ a fin de traer el punto $z = \infty$ a la parte finita de la esfera. En esta nueva carta, $f$ se representa como $g(w) = f(1/z)$. Entonces decimos que $f$ tiene una singularidad removible, un polo de orden $m$, o una singularidad esencial en $z = \infty$, si $g$ tiene una singularidad del tipo correspondiente en $w = 0$.
\begin{enumerate}[label=(\alph*)]
    \item Pruebe que toda función entera con una singularidad removible en $\infty$ es constante.
    \item Pruebe que toda función entera con un polo de orden $m$ en $\infty$ es un polinomio de grado $m$.
    \item Caracterice las funciones racionales con una singularidad removible en el infinito.
    \item Caracterice las funciones racionales con un polo de orden $m$ en el infinito.
\end{enumerate}
\end{exercise}

\begin{solution}
Trataremos a las singularidades removibles como polos de orden cero.
\begin{enumerate}[label=(\alph*)]
    \item Sea $f(z)$ un función entera. Las siguientes proposiciones son equivalentes:
    \begin{itemize}
        \item $f(z)$ tiene una singularidad removible en el infinito.
        \item $f(z)$ es acotada en una vecindad agujereada del infinito.
        \item $f(z)$ es acotada globalmente, porque la esfera de Riemann es compacta.
        \item $f(z)$ es constante, porque es entera y acotada.
    \end{itemize}
    
    \item Sea $f(z)$ un función entera. Las siguientes proposiciones son equivalentes:
    \begin{itemize}
        \item $f(z)$ tiene un polo de orden $m \in \Z^+$ en el infinito.
        \item $f(z) / z^m$ converge a algún $b \in \C^\star$ cuando $z \to \infty$.
        \item $f(z) / z^m - b$ converge a cero cuando $z \to \infty$.
        \item $f(z) - bz^m$ tiene un polo de orden menor que $m$ en el infinito.
        \item $f(z) - bz^m$ es un polinomio de grado menor que $m$, inductivamente.
        \item $f(z)$ es un polinomio de grado $m$.
    \end{itemize}
    
    El caso base para la inducción es el ítem a).
    
    \item Sea $f(z) = p(z) / q(z)$ una función racional. Las siguientes proposiciones son equivalentes:
    \begin{itemize}
        \item $f(z)$ tiene una singularidad removible en el infinito.
        \item $f(z)$ es acotada en una vecindad agujereada del infinito.
        \item $p(z)$ no excede en grado a $q(z)$.
    \end{itemize}
    
    \item Sea $f(z) = p(z) / q(z)$ una función racional. Las siguientes proposiciones son equivalentes:
    \begin{itemize}
        \item $f(z)$ tiene un polo de orden $m \in \Z^+$ en el infinito.
        \item $f(z) / z^m$ converge a algún $b \in \C^\star$ cuando $z \to \infty$.
        \item $f(z) / z^m - b$ converge a cero cuando $z \to \infty$.
        \item $f(z) - bz^m$ tiene un polo de orden menor que $m$ en el infinito.
        \item $p(z) - bz^m$ excede en grado a $q(z)$ por una cantidad menor que $m$, inductivamente.
        \item $p(z)$ excede en grado a $q(z)$ por $m$.
    \end{itemize}
    
    El caso base para la inducción es el ítem c).
\end{enumerate}
El procedimiento para estudiar la singularidad de una función racional en el infinito es
\begin{itemize}
    \item Utilizar la división de polinomios para expresar la función racional como
    $$f(z) = p(z) + \frac {r(z)} {q(z)}$$
    donde $r(z)$ tiene grado menor que $q(z)$.
    
    \item Por construcción, el segundo término tiene una singularidad removible en el infinito, porque
    $$\lim_{z \to \infty} \frac {r(z)} {q(z)}$$
    tiene un valor finito.
    
    \item Por construcción, $f(z)$ y $p(z)$ tienen el mismo tipo de singularidad en el infinito:
    \begin{itemize}
        \item La singularidad es removible si $p(z)$ es constante.
        \item La singularidad es un polo de orden $m$ si $p(z)$ tiene grado $m \in \Z^+$.
    \end{itemize}
    
    \item La singularidad nunca es esencial.
\end{itemize}
Las funciones racionales son precisamente los endomorfismos analíticos de la esfera de Riemann.
\end{solution}
